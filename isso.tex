%%%%%%%%%%%%%%%%%
% This is an example CV created using altacv.cls (v1.1.5, 1 December 2018) written by
% LianTze Lim (liantze@gmail.com), based on the
% Cv created by BusinessInsider at http://www.businessinsider.my/a-sample-resume-for-marissa-mayer-2016-7/?r=US&IR=T
%
%% It may be distributed and/or modified under the
%% conditions of the LaTeX Project Public License, either version 1.3
%% of this license or (at your option) any later version.
%% The latest version of this license is in
%%    http://www.latex-project.org/lppl.txt
%% and version 1.3 or later is part of all distributions of LaTeX
%% version 2003/12/01 or later.
%%%%%%%%%%%%%%%%

%% If you are using \orcid or academicons
%% icons, make sure you have the academicons
%% option here, and compile with XeLaTeX
%% or LuaLaTeX.
% \documentclass[10pt,a4paper,academicons]{altacv}

%% Use the "normalphoto" option if you want a normal photo instead of cropped to a circle
% \documentclass[10pt,a4paper,normalphoto]{altacv}

\documentclass[10pt,a4paper,ragged2e]{altacv}

%% AltaCV uses the fontawesome and academicon fonts
%% and packages.
%% See texdoc.net/pkg/fontawecome and http://texdoc.net/pkg/academicons for full list of symbols. You MUST compile with XeLaTeX or LuaLaTeX if you want to use academicons.

% Change the page layout if you need to
\geometry{left=1cm,right=9cm,marginparwidth=6.8cm,marginparsep=1.2cm,top=1.25cm,bottom=1.25cm}

% Change the font if you want to, depending on whether
% you're using pdflatex or xelatex/lualatex
\ifxetexorluatex
  % If using xelatex or lualatex:
  \setmainfont{Carlito}
\else
  % If using pdflatex:
  \usepackage[utf8]{inputenc}
  \usepackage[T1]{fontenc}
  \usepackage[default]{lato}
\fi

% Change the colours if you want to

%red
%\definecolor{RedSpecial}{HTML}{AA4653}
%\definecolor{SlateGrey}{HTML}{2E2E2E}
%\definecolor{LightGrey}{HTML}{666666}
%\colorlet{heading}{RedSpecial}
%\colorlet{accent}{RedSpecial}
%\colorlet{emphasis}{SlateGrey}
%\colorlet{body}{LightGrey}

%blue
\definecolor{BlueSpecial}{HTML}{2B7A78}
\definecolor{SlateGrey}{HTML}{2E2E2E}
\definecolor{LightGrey}{HTML}{666666}
\colorlet{heading}{BlueSpecial}
\colorlet{accent}{BlueSpecial}
\colorlet{emphasis}{SlateGrey}
\colorlet{body}{LightGrey}


%idioma
\usepackage[spanish,es-lcroman]{babel}
% Change the bullets for itemize and rating marker
% for \cvskill if you want to
\renewcommand{\itemmarker}{{\small\textbullet}}
\renewcommand{\ratingmarker}{\faCircle}


\begin{document}
\name{Javier Conde}
\tagline{Ingeniero de Telecomunicaciones, UPM}
\photo{4cm}{photo}
\personalinfo{%
  % Not all of these are required!
  % You can add your own with \printinfo{symbol}{detail}
  \anniversary{22/10/1996}
  \email{4b3javi@gmail.com}
  \location{Madrid, España}\\
%  \phone{xxxxxxxxxx}
  \github{github.com/javicond3}
  \linkedin{linkedin.com/in/javier-conde-diaz}
    
%   \orcid{orcid.org/0000-0000-0000-0000} % Obviously making this up too. If you want to use this field (and also other academicons symbols), add "academicons" option to \documentclass{altacv}
}

%% Make the header extend all the way to the right, if you want.
\begin{fullwidth}
\makecvheader
\end{fullwidth}


\AtBeginEnvironment{itemize}{\small}


\cvsection[page1sidebar]{Formación Académica}

\cvevent{Doctorado en Ingeniería de Sistemas Telemáticos}{Universidad Politécnica de Madrid}{2020 -- Actualidad}{Madrid}

\cvevent{Máster Universitario en Ingeniería de Telecomunicación {\small(Media 9.5, MH 10 asignat.)}}{Universidad Politécnica de Madrid}{2018 -- 2020}{Madrid}
\cvevent{Grado en Ingeniería de Tecnologías y Servicios de Telecomunicación {\small(Media 8.8, 1º en especialidad de telemática, MH 21 asignat.)}}{Universidad Politécnica de Madrid}{2014 -- 2018}{Madrid}

%%\cvevent{Bachillerato científico-tecnológico %%\textit{\small(Media 10, 1º de la promoción)}}{IES Eduardo %%Blanco Amor}{2012 -- 2014}{Ourense}

\divider

\cvsection[page1sidebar]{Experiencia laboral}

\cvevent{Grupo Internet Nueva Generación}{Departamento de Ingeniería de Sistemas Telemáticos, UPM}{Enero 2020 -- Actualidad}{Madrid}
\begin{itemize}
\item Participación en proyectos europeos en el área de Linked Open Data, Machine Learning y Digital Twins. 
\item Contribución en FIWARE: The Open Source Platform for Our Smart Digital Future.
\end{itemize}
\divider

\cvevent{Colaboraciones Docentes}{UPM}{Enero 2019 -- Actualidad}{Madrid}
\begin{itemize}
\item Dirección de Trabajos de Fin de Titulación.   
\end{itemize}
\divider

\cvevent{Integración Digital de la ETSIT}{Beca Fundación Ángel Barbero Martín de Vidales}{Marzo 2018 -- Julio 2020}{Madrid}
\begin{itemize}
\item Desarrollo y operación de aplicaciones web para la universidad.   
\item Mantenimiento y gestión de servidores. 
\item Diseño, mantenimiento y explotación de bases de datos.
\end{itemize}
\divider

\cvevent{SENER Ingeniería y Construcción}{Beca Universitaria}{Enero 2018 -- Marzo 2018}{Madrid}
\begin{itemize}
\item Sector aerospacial.
\item Despliegue de un servidor para el tratamiento de información de medida de equipos y detección de fallos.  
\end{itemize}
\divider


\clearpage

\end{document}
